\documentclass[12pt]{article}
\usepackage[dvipsnames]{xcolor}
\usepackage[T1]{fontenc}
\usepackage{mathtools}
\usepackage[french]{babel}
\usepackage{amsmath,amssymb,amsthm}
\usepackage{framed}
\usepackage{lmodern}
\usepackage{utils}
\usepackage{pdfpages}
\usepackage{irif}
\usepackage{listings}
\usepackage{listingsutf8}
\hyphenpenalty 10000
\exhyphenpenalty 10000
\definecolor{codegreen}{rgb}{0,0.6,0}
\definecolor{codegray}{rgb}{0.5,0.5,0.5}
\definecolor{codepurple}{rgb}{0.58,0,0.82}
\definecolor{backcolour}{rgb}{0.96,0.96,0.95}
\newcommand*{\pmZpmZ }{p^m\Z \x p^m\Z}
\newcommand*{\ZZpmZ}{\Z^2/\pmZpmZ}
\newcommand*{\ZZnZ}{\Z^2/\nZnZ}
\newcommand{\ZpmZ}{\Z/p^m\Z}
\newcommand{\ZZpm}{\ZpmZ \x \ZpmZ}
\newcommand{\nZnZ}{n\Z \x n\Z}
\newcommand{\Mz}{\Set*{
	\begin{pmatrix}
		p^a  & 0   \\
		j & p^b
	\end{pmatrix}
}{
	\begin{aligned}
		 & (a, b) \in A_0      \\
		 & 0 \le j < p^{b}
	\end{aligned}
}}
\newcommand{\Mk}{\Set*{
	\begin{pmatrix}
		p^a  & 0   \\
		jp^k & p^b
	\end{pmatrix}
}{
	\begin{aligned}
		 & (a, b) \in A_k      \\
		 & 0 \le j < p^{b - k}
	\end{aligned}
}}
\newcommand{\Az}{\Set*{(a,b)}{ a + b \le m}}
\newcommand{\Ak}{\Set*{
	(a,b)}{
	\begin{aligned}
		 & a \le m, b \le m \\
		 & a + b = m + k
	\end{aligned}
}}

\lstdefinestyle{mystyle}{
    backgroundcolor=\color{backcolour},
    commentstyle=\color{codegreen},
    keywordstyle=\color{magenta},
    numberstyle=\tiny\color{codegray},
    stringstyle=\color{codepurple},
    breakatwhitespace=false,
    breaklines=true,
    captionpos=b,
    keepspaces=true,
    numbers=left,
    numbersep=5pt,
    showspaces=false,
    showstringspaces=false,
    showtabs=false,
	extendedchars=true,
    tabsize=4,
	basicstyle=\ttfamily,
  	mathescape,
	inputencoding=utf8/latin1,
	literate=%
		{é}{{\'e}}{1}%
		{è}{{\`e}}{1}%
		{à}{{\`a}}{1}%
		{ç}{{\c{c}}}{1}%
		{œ}{{\oe}}{1}%
		{ù}{{\`u}}{1}%
		{É}{{\'E}}{1}%
		{È}{{\`E}}{1}%
		{À}{{\`A}}{1}%
		{Ç}{{\c{C}}}{1}%
		{Œ}{{\OE}}{1}%
		{Ê}{{\^E}}{1}%
		{ê}{{\^e}}{1}%
		{î}{{\^i}}{1}%
		{ô}{{\^o}}{1}%
		{û}{{\^u}}{1}%
		{ë}{{\¨{e}}}1
		{û}{{\^{u}}}1
		{â}{{\^{a}}}1
		{Â}{{\^{A}}}1
		{Î}{{\^{I}}}1
}

\lstset{style=mystyle}

\begin{document}
\includepdf{title.pdf}
\pagenumbering{arabic}
\hfill
\thispagestyle{empty}
\newpage
\tableofcontents
%%%%%%%%%%%%%%%%%%%%%%%%%%%%%%%%%%%%%%%%%%%%%%%%%%%%%%%%%%%%%%%%%%%%%%%%
\newpage
\section{Introduction}
Il est très facile de décrire tous les sous-groupes d'un groupe cyclique
d'ordre $n$ : il y en a exactement un par diviseur positif de $n$.
Pourtant, étonnamment, décrire tous les sous-groupes d'un groupe abélien
est en général un problème difficile.
Dans ce projet, nous nous proposons de considérer cette question pour le groupe $\ZZ$.

D'un point de vue théorique, nous mettrons en avant la génération et la
caractérisation des sous-groupes grâce aux vecteurs colonnes des matrices à coefficients entier et en particulier
aux formes normales de Hermite. Nous montrerons aussi une formule permettant de les compter.

D'un point de vue pratique, nous créerons un programme \textsc{OCaml} capable de générer les
sous-groupes de $\ZZ$ ainsi que leur treillis à partir d'un entier donné en paramètres.

%%%%%%%%%%%%%%%%%%%%%%%%%%%%%%%%%%%%%%%%%%%%%%%%%%%%%%%%%%%%%%%%%%%%%%%%
\newpage
\section{Quelques simplifications du problème}
%- - - - - - - - - - - - - - - - - - - - - - - - - - - - - - - - - - - -
\subsection{Décomposition de $n$ en éléments irréductibles}\label{theoreme_chinois}

Nous pouvons tout d'abord simplifier le problème aux cas où $n = p^m$ avec $p$ un nombre premier
et $m \in \N$. En effet la proposition suivante nous garantit que le résultat est isomorphe
\begin{proposition}
	Soit $n = \prod\limits_{i = 1}^k p_i^{\alpha_i}$, avec $p_i$ des nombres premiers, alors
	$$(\ZZ) \isom \prod_{i = 1}^k(\Z/p_i^{\alpha_i}\Z)^2$$
\end{proposition}

\begin{proof}
	Soit $n = \prod\limits_{i = 1}^k p_i^{\alpha_i}$. Par le théorème des restes chinois, nous avons
	$$ \ZnZ \isom (\Z/p_1^{\alpha_1}\Z) \x \cdots \x (\Z/p_k^{\alpha_k}\Z)$$
	En particulier,
	\begin{equation*}
		\begin{split}
			\ZZ & \isom
			(\Z/p_1^{\alpha_1}\Z) \x \cdots \x (\Z/p_k^{\alpha_k}\Z) \x (\Z/p_1^{\alpha_1}\Z)%
			\x \cdots \x (\Z/p_k^{\alpha_ik}\Z)\\
			& \isom (\Z/p_1^{\alpha_1}\Z)^2 \x \cdots \x (\Z/p_k^{\alpha_k}\Z)^2
		\end{split}
	\end{equation*}
\end{proof}

En pratique, pour décomposer en entier en facteurs irréductibles, nous avons utilisé\\
la procédure de \textsc{$\rho$-Pollard} pour obtenir un diviseur de $n$:
\begin{lstlisting}
fonction rho_pollard P n x y k i d
    Si d <> 1:
        Retourne d
    Sinon:
        x = P(x) mod n
        d = pgcd(|y - x|, n)
        Si i = k:
            Alors, Retourne rho_pollard loop P n x x 2k (i + 1) d
        Sinon Retourne rho_pollard P n x y k (i + 1) d
\end{lstlisting}
Puis nous répétons la procédure jusqu'à que les diviseurs soient premier.\\
En triant et en regroupant les nombres premier, nous obtenons donc les différents $p^{\alpha_i}_i$.\\
Dans notre implémentation, $P(X) = X^2 - 1$ et $n$ n'est pas premier.
%TODO : Ajout algo test primarite ?
%- - - - - - - - - - - - - - - - - - - - - - - - - - - - - - - - - - - - - -
\newpage
\subsection{Simplification des sous-groupes}\label{simp_ss_gr}

\begin{proposition}
	$$\Z^2/n\Z \x n\Z \isom \ZZ $$
\end{proposition}
\begin{proof}
	Soit \app{\varphi}{\Z^2}{\ZZ}{(a,b)}{(\bar a, \bar b)}
	$\varphi$ est surjectif par définition de la classe d'équivalence de a et b.
	Montrons que $\ker \varphi = n\Z \x n\Z$.

	\begin{equation*}
		\begin{split}
			&(a,b) \in \ker \varphi \\
			&\text{ssi } \varphi(a,b) = (\bar 0, \bar 0)\\
			&\text{ssi } (\bar a, \bar b) = (\bar 0, \bar 0)\\
			&\text{ssi } \bar a = \bar 0 \text{ et } \bar b = \bar 0\\
			&\text{ssi } a \in n\Z \text{ et } b \in n\Z\\
			&\text{ssi } (a,b) \in  n\Z \x n\Z
		\end{split}
	\end{equation*}
	Ainsi par le premier théorème d'isomorphisme, nous avons
	$$\Z^2/n\Z \x n\Z \isom \ZZ $$
\end{proof}
Ainsi le problème se résout à trouver les sous-groupes $G$ de $\Z^2$ tels que
$H = \matsqr{a}{c}{b}{d}$\\
et
$n\Z \x n\Z \subseteq G = \gen{\vectcolsqr{\bar a}{\bar b}, \vectcolsqr{\bar c}{\bar d}}$

%%%%%%%%%%%%%%%%%%%%%%%%%%%%%%%%%%%%%%%%%%%%%%%%%%%%%%%%%%%%%%%%%%%%%%%%%%%%%%
\newpage
\section{Matrices à coefficients entier et forme normales de Hermite}
Nous avons vu dans la section précédente qu'il était possible de caractériser les sous-groupe de
$\ZZ$ par une matrice $H = \matsqr{a}{c}{b}{d}$. Cependant, ces matrices ne sont pas uniques. C'est
pourquoi, nous allons utiliser les formes normales d'Hermite.
Énonçons d'abord une proposition sur les matrices à coefficients entier qui nous sera fort
utile par la suite.

%- - - - -- - - - - - - - - - - - - - - - - - - - - - - - - - - - - - - - -
\subsection{Matrices à coefficients entier}
\begin{proposition}\label{ima_imaq}
	Soient $A \in \M_{m,n}(\Z)$ et $Q \in \GL_n(\Z)$, alors
	$$\im AQ = \im A$$
\end{proposition}
\begin{proof}
	Soit $y \in \im AQ$, il existe $x \in \Z^n$ tel que $y = AQx$. Or,
	\begin{align*}
		         & y = AQx     \\
		\implies & y = A(Qx)   \\
		\implies & y \in \im A
	\end{align*}
	Donc, $\im AQ \subseteq \im A$.
	Soit $y \in \im A$. Il existe $x \in \Z^n$ tel que $y = Ax$.\\
	Cherchons $z \in \Z^n$ tel que $y = Ax = AQz$
	\begin{align*}
		         & Ax = AQz                                     \\
		\implies & A(x) = A(Qz)                                 \\
		\implies & x = Qz                                       \\
		\implies & \inv Q x = z \text{ (car $Q \in \GL_n(\Z)$)}
	\end{align*}
	Donc, il existe bien un $z \in \Z^n$ tel que $AQz = y$. Donc, $y \in \im AQ$.\\
	D'où $\im AQ = \im A$

\end{proof}

%- - - - - - - - - - - - - - - - - - - - - - - - - - - - - - - - - - - - -
\newpage
\subsection{Formes normales de Hermite}
Nous allons désormais énoncer la définition de la forme normale de Hermite.
\begin{definition}
	Soit $A \in \M_{m,n}(\Z)$. Alors, il existe une unique matrice échelonnée
	réduite suivant les colonnes $H \in \M_{m,n}(\Z)$ telle qu'il existe $Q \in \GL_n(\Z)$
	avec $H = AQ$. La matrice $H$ s'appelle la forme normale de Hermite de A.
\end{definition}

\begin{proof}
	Nous admettons l'unicité.\\
	L'algorithme suivant nous montre son existence.
	\begin{lstlisting}
Fonction hermite_aux($A$, $i$):
	Pour chaque $j$ allant de $i$ à $m$ :
		Si i = j :
			Si $a_{ij} < 0$ : réaliser l'opération $C_j \leftarrow -C_j$
		Sinon :
			Si $a_{ij}$ < 0 : réaliser l'opération $C_j \leftarrow -C_j$
			$k,r$ = div_euclide($a_{ij}$, $a_{ii}$)
			réaliser l'opération $C_j \leftarrow C_j - kC_i$
	Si $\forall i < j <= m, a_{ij} = 0$ :
		Réduire à gauche du pivot
		Retourner A
	Sinon
	$d_{k} = \min(\Set*{a_{ij}}{i <= j <= n a_{ij} \ne 0})$
	Permuter $C_k$ avec $C_i$
	hermite_aux($A$, $i$)

Fonction hermite(A) :
	Pour chaque $i$ allant de A à $n$ :
		$d_{k} = \min(\Set*{a_{ij}}{i <= j <= n a_{ij} \ne 0})$
		Si $d = None$ :
			Continuer boucle
		Sinon :
			Permuter $C_k$ avec $C_i$
			$A$ = hermite_aux($A$, $i$)
	vérifier signe des pivots de $A$
	Retourner $A$

\end{lstlisting}
	\newpage\noindent
	Montrons la terminaison de l'algorithme.\\
	La fonction \texttt{hermite\_aux} se repose sur l'algorithme d'Euclide.\\
	En effet, pour tout $i < j < m$, nous réalisons la division euclidienne de $a_{ij}$ par $a_{ii}$.\\
	Si à la fin de la boucle il existe $j$ tel que $a_{ij} < a_{ii}$, alors nous recommençons
	en permutant $C_j$ et $C_i$ et $a_{ij}$ devient	notre nouveau pivot.\\
	Ainsi, par la correction de l'algorithme d'Euclide, il existe un rang $N$ où tous les
	$a_{ij}$ avec $j> i$ sont tous nuls. Ainsi la fonction \texttt{hermite\_aux} se termine.
	La fonction \texttt{hermite} étant	seulement une boucle, elle se termine également.
	Donc, l'algorithme se termine bien.\\

	\noindent
	Montrons la correction de l'algorithme.\\
	La fonction hermite\_aux se repose sur l'algorithme
	d'Euclide en utilisant des opérations élémentaires sur les matrices.
	Par la correction de l'algorithme d'Euclide, nous pouvons en déduire que pour
	tout $j > i$, $a_{ij} = 0$.\\
	De plus, avant de retourner, nous réalisons la division euclidienne des $a_{ij}$ par
	$a_{ii}$ avec $0 < j < i$.\\
	Donc, les $a_{ij}$ sont les restes des divisions euclidiennes et sont donc réduit au maximum.\\
	Nous réalisons ces opération sur toutes les lignes sans jamais revenir sur les lignes précédentes.
	Enfin, nous vérifions le signe de pivot et nous changeons la colonne de signe si nécessaire.\\
	Ainsi la matrice obtenue est bien échelonnée réduite,
	il s'agit donc d'une forme normale de Hermite, ce qui prouve donc son existence.

\end{proof}

\begin{example}
	\begin{equation*}
		\begin{split}
			A =
			\begin{pmatrix}
				2  & 1  \\
				4  & 10 \\
				5  & 13 \\
				13 & 12
			\end{pmatrix}
			\overset{C_2 \leftrightarrow C_1}{\longrightarrow}
			\begin{pmatrix}
				1  & 2  \\
				10 & 4  \\
				13 & 5  \\
				12 & 13
			\end{pmatrix}
			\overset{C_2 \leftarrow C_2 - 2C_1}{\longrightarrow}
			\begin{pmatrix}
				1  & 0   \\
				10 & -16 \\
				13 & 3   \\
				12 & -14
			\end{pmatrix}
			\overset{C_2 \leftarrow - C_2}{\longrightarrow}
			\begin{pmatrix}
				1  & 0  \\
				10 & 16 \\
				13 & -3 \\
				12 & 14
			\end{pmatrix} = H
		\end{split}
	\end{equation*}

\end{example}
\begin{remark}
	Il existe des algorithmes beaucoup plus efficace pour calculer la forme normale de Hermite comme
	l'algorithme de de Domich \& Ai (1989) qui réalise les calculs modulo le déterminant de $A$ ou
	l'algorithme de Micciancio-Warinshi. Cependant, ces algorithmes étant plus ou moins compliqué,
	le choix ici a été de faire nous-mêmes un algorithme à partir de la méthode naïve employée
	lors du calcul de la forme normale de Hermite à la main.
\end{remark}

\newpage
\begin{definition}
	Soient $A \in \M_{m,n}$, $B \in \M_{m,p}$ deux formes normales de Hermite
	$$A \sim B \text{ ssi les colonnes non nulles de } A
		\text{ sont les mêmes que les colonnes non nulles de } B.$$
\end{definition}
Nous allons énoncer quelques résultats utiles des formes normales de Hermite.
Tout d'abord, la proposition suivante nous permet de nous restreindre à certaines matrices sous une forme particulière
dont les colonnes génèrent un sous groupe de $\Z^2$
\begin{proposition}\label{ima_imh}
	Soit $A \in \M_{m,n}(\Z)$ et soit $H$ sa forme normale d'Hermite. Alors,
	$$\im A = \im H$$
\end{proposition}

\begin{proof}
	C'est une application de la proposition \ref{ima_imaq} avec $H= AQ$ avec $Q \in \GL_n(\Z)$

\end{proof}

Ainsi les colonnes de la forme normale de Hermite $H$ d'une matrice $A$ génèrent le même
sous-groupe que les colonnes de $A$.\\
Nous n'avons donc plus qu'à trouver des matrices de la forme
$H = \matsqr{a}{0}{b}{c}$ telle que
$$n\Z \x n\Z \subseteq \gen{\vectcolsqr{\bar a}{\bar b},\vectcolsqr{\bar 0}{\bar c}}$$

\begin{proposition}\label{ha_hb_ssi_ima_imb}
	Soient $A,B \in \M_{m,n}(\Z)$.\\
	Soit $H_A$ (resp. $H_B$) la forme normale de Hermite de $A$ (resp. $B$). Alors,
	$$ H_A = H_B \ssi \im A = \im B$$
\end{proposition}

\begin{proof}
	Supposons que $H_A = H_B$.	Par la proposition \ref{ima_imh}, nous avons
	$$\im A = \im H_A = \im H_B = \im B$$
	Réciproquement, supposons que $\im A = \im B$. Par la proposition \ref{ima_imh}, nous avons
	$$\im A = \im H_A = \im H_B = \im B$$
	Donc, les vecteurs colonnes de $H_A$ qui génèrent $\im H_A$ sont les mêmes que ceux de
	$H_B$ qui génèrent $\im H_B$, d'où $H_A = H_B$

\end{proof}

\newpage
Cette proposition nous affirme donc qu'en traitant seulement les formes normales de \\
Hermite, nous pourrons générer tous les sous-groupes de $\Z^2$

La proposition suivante, va nous être utile pour trouver la bonne forme des matrices de Hermite
ainsi que la génération du treillis.

\begin{proposition}\label{guh_g_ssi_hab_sim_b}
	Soient $A,B \in \M_{m,n}$ deux formes normales de Hermite et soit $G$ (resp. $H$) le
	groupe engendré par les colonnes de $A$ (resp. $B$). Alors,
	$$G \cup H = G \ssi hermite(A | B) \sim A$$
	où $hermite(A|B)$ est la forme normale de hermite de la matrice augmentée $(A | B)$.
\end{proposition}

\begin{proof}
	Supposerons que $G \cup H = G$. Alors, $ H \subseteq G$.Donc
	$$\forall x \in H \cma  \ex y \in G \cma \ex \lambda \in \Z \cma x = \lambda y$$
	En particulier la base de $H$ est une combinaison linéaire de la base de $G$. D'où
	$$hermite(A | B) = (A | 0) \sim A$$
	\noindent
	Réciproquement, soit $hermite(A | B) \sim A$. Alors, par définition de la relation
	d'équivalence,\\
	$hermite(A | B) = (A | 0)$. Ainsi les colonnes de B sont des combinaisons linéaire des
	colonnes de $A$,\\
	\cad \, la base de $H$ est une combinaison linéaire de la base de $G$.
	$$\text{D'où } H \subseteq G \text{ et } G \cup H = G$$
\end{proof}

Nous avons désormais tous les outils à notre disposition pour
démontrer la forme voulue des matrices ainsi que la formule permettant de compter le nombre
de sous-groupe de $\ZZnZ$

%%%%%%%%%%%%%%%%%%%%%%%%%%%%%%%%%%%%%%%%%%%%%%%%%%%%%%%%%%%%%%%%%
\newpage
\section {Génération et énumération des sous-groupes}
Nous allons voir dans cette section la forme des matrices dont les vecteurs colonnes génèrent
les sous-groupes de $\ZZnZ$, la formule permettant de les compter ainsi que quelques propositions
sur leurs caractéristiques. Nous avons montré dans\\
la section \ref{theoreme_chinois} que nous
pouvons nous restreindre aux cas où $n = p^m$ avec $p$ un nombre premier et $m \in \N$.\\
Ainsi dans cette section, nous supposerons que $n = p^m$.

%- - - - - - - - - - - - - - - - - - - - - - - - - - - - - - - -
\subsection{Génération des sous-groupes}
Commençons tout d'abord par montrer la forme des matrices dont les vecteurs colonnes génèrent
les sous-groupes de $\ZZpmZ$. Dans la section \ref{simp_ss_gr}, nous avons montré que les
sous-groupes de $\Z^2$ recherché sont les sous groupes $G$ tels que
$\pmZpmZ \subseteq G$, \cad
$$ G \cup \pmZpmZ = G$$
Par la proposition \ref{guh_g_ssi_hab_sim_b}, cela revient à trouver les
matrices $H = \begin{pmatrix}
		\alpha & 0      \\
		\beta  & \gamma
	\end{pmatrix}$
telles que $$(H | Mat(\pmZpmZ)) =
	\begin{pmatrix}
		\alpha & 0      & p^m & 0   \\
		\beta  & \gamma & 0   & p^m
	\end{pmatrix}
	\sim	\begin{pmatrix}
		\alpha & 0      \\
		\beta  & \gamma
	\end{pmatrix}
	= H
$$

\begin{theorem}
	Les seules matrices dont les colonnes génèrent un sous-groupe de $\ZZpmZ$
	sont les matrices de la forme
	$$H =
		\begin{pmatrix}
			p^a & 0   \\
			j   & p^b
		\end{pmatrix}
		\text{avec $a \le m$, $b \le m$ et $j < p^b$}
	$$
	$$ \text{ ou }$$
	$$ H =\begin{pmatrix}
			p^a  & 0   \\
			jp^k & p^b
		\end{pmatrix}
		\text{avec $a \le m$, $b \le m$, $k \le m$ et $j < p^{b - k}$}
	$$

\end{theorem}
\begin{proof}
	Montrons tout d'abord que ces matrices sont les seules qui génèrent les sous-groupe de $\ZZ$.\\
	Supposons qu'il existe $H,H'$ deux formes normales de Hermite qui génèrent $G$ un sous-groupe
	de $\ZZ$. Alors, nous avons $\im H = \im H'$ et par la proposition \ref{ha_hb_ssi_ima_imb}, $H = H'$.\\
	Nous pouvons donc créer une classe d'équivalence pour la relation $\sim$ où la forme normale
	de hermite est la représentante de ces classes.\\
	\newpage
	\noindent
	Montrons désormais l'existence de telles matrices.\\
	Soit
	$H = \begin{pmatrix}
			a & 0 \\
			b & c
		\end{pmatrix}$
	Nous cherchons à réaliser des opérations élémentaires telles que
	\begin{equation*}
		A = \begin{pmatrix}
			\alpha & 0      & p^m & 0   \\
			\beta  & \gamma & 0   & p^m
		\end{pmatrix}
		\longrightarrow
		\begin{pmatrix}
			\alpha & 0      & 0 & 0 \\
			\beta  & \gamma & 0 & 0
		\end{pmatrix}
	\end{equation*}
	Annuler $a_{1\,3} = a_{2\,4} = p^m$ nécessite les conditions suivante sur $\alpha$ et $\beta$ :
	$$\begin{cases*}
			\alpha = p^a \text{ avec } a \le m\\
			\gamma = p^b \text{ avec } a \le m\\
		\end{cases*}$$
	Nous pouvons donc réaliser les opérations $C_3 \leftarrow C_3 -p^{m-a}C_1$ et
	$C_4 \leftarrow C_4 -p^{m-b}C_2$. Ceci nous donne donc :
	\begin{equation*}
		\begin{pmatrix}
			\alpha & 0      & p^m & 0   \\
			\beta  & \gamma & 0   & p^m
		\end{pmatrix}
		\overset{C_3 \leftarrow C_3 -p^{m-a}C_1 }%
		{\overset{ C_4 \leftarrow C_4 -p^{m-b}C_2}{\longrightarrow}}
		\begin{pmatrix}
			\alpha & 0      & 0             & 0 \\
			\beta  & \gamma & -p^{m-a}\beta & 0
		\end{pmatrix}
	\end{equation*}
	Nous cherchons désormais $\beta$ tel que $\divided{p^b}{\beta p^{m-a}}$.\\
	Tout d'abord, par la définition de la forme normale de Hermite, $\beta< p^b$.\\
	Nous pouvons isoler deux cas en fonction de $a$ et $b$ :\\
	\begin{itemize}
		\item Si $a + b \le m$, alors $b \le m - a$ et $\divided{p^b}{p^{m-a}}$ et
		      donc $\divided{p^b}{\beta p^{m-a}}$.\\
		\item Sinon $m \le a + b \le 2m \implies 0 \le a + b -m \le n$.\\
		      Nous posons $k = a + b -m$ et nous avons donc $0 \le k \le m$.\\
		      Nous posons $\beta = ip^k$ avec $0 \le i < p^{b - k}$. Nous avons bien
		      $\fa i \cma \beta = ip^k < p^b$.\\
		      De plus, $\divided{p^b}{ip^kp^{m-a}}$ car  $\divided{p^b}{p^kp^{m-a}}$
		      car  $b = k + m - a$
	\end{itemize}
	Ainsi, dans les deux cas, nous pouvons annuler $a_{2\,3}$ et nous obtenons bien la matrice
	suivante en faisant l'opérations élémentaire.
	$\begin{pmatrix}
			\alpha & 0      & 0 & 0 \\
			\beta  & \gamma & 0 & 0
		\end{pmatrix}$.\\
	Dans le premier cas, nous obtenons en posant $j = \beta < p^b$
	\begin{equation*}
		\begin{pmatrix}
			p^a & 0   & 0 & 0 \\
			j   & p^b & 0 & 0
		\end{pmatrix}
		\sim
		\begin{pmatrix}
			p^a & 0   \\
			j   & p^b
		\end{pmatrix}
	\end{equation*}
	Dans le deuxième cas, nous obtenons en posant $j = i < p^{b-k}$
	\begin{equation*}
		\begin{pmatrix}
			p^a  & 0   & 0 & 0 \\
			jp^k & p^b & 0 & 0
		\end{pmatrix}
		\sim
		\begin{pmatrix}
			p^a  & 0   \\
			jp^k & p^b
		\end{pmatrix}
	\end{equation*}
	\noindent
	Ce qui démontre bien l'existence de ces matrices.

\end{proof}

\begin{corollary}\label{union_m}
	Soit la suite $\suit{A}{k}{0 \le k \le n}$ telle que
	\begin{equation*}
		\begin{split}
			&A_0 = \Set*{(a,b)}{ a + b \le m}\\
			&A_k = \Set*{
				(a,b)}{
				\begin{aligned}
					 & a \le m, b \le m \\
					 & a + b = m + k
				\end{aligned}
			}
		\end{split}
	\end{equation*}
	Alors, l'ensemble des matrices du théorème, \cad, les matrices dont les colonnes
	génèrent les sous-groupes de $\ZZpmZ$ est
	$$M = \bigsqcup_{k = 0}^mM_k$$
	où
	\begin{equation*}
		M_k = \Set*{
			\begin{pmatrix}
				p^a  & 0   \\
				jp^k & p^b
			\end{pmatrix}
		}{
			\begin{aligned}
				 & (a, b) \in A_k      \\
				 & 0 \le j < p^{b - k}
			\end{aligned}
		}
	\end{equation*}
\end{corollary}

\begin{proposition}
	Si $A \in  \M_2$ est injective et $B = AC$ avec $C \in \M_2$ alors,
	$$\im A/\im B  \isom \Z^2/\im C$$
\end{proposition}

\begin{proof}
	Considérons \app{f}{\Z^2}{\im A \rightarrow \im A / \im B}{x}{Ax \;\;\;\mapsto Ax + \im B}
	Cette application est surjective par définition. Calculons sont noyau.
	\begin{equation*}
		\begin{aligned}
			x \in Ker f & \ssi Ax \in \im B             \\
			            & \ssi \ex y \cma Ax = by = ACy \\
			            & \ssi x \in \im C
		\end{aligned}
	\end{equation*}
	D'où par le premier théorème d'isomorphisme,
	$$\im A/\im B  \isom \Z^2/\im C$$
\end{proof}
\newpage

\begin{proposition}
	Soient $A,B \in \M_2$
	$$\im B \subset \im A \implies \ex C \in \M_2 \cma B = AC$$
\end{proposition}

\begin{proof}
	Notons que par la proposition \ref{ima_imh}, on peut supposer sans pertes de généralités que $A$ et $B$
	sont des formes normales de Hermite.\\
	Comme $\im A$ (resp. $\im B$) est le groupe généré par les colonnes de $A$ (resp. $B$)
	et que $\im A \cup \im B = \im A$ car $\im B \subset \im A$,
	par la proposition \ref{guh_g_ssi_hab_sim_b}, on a
	$$\im A \cup \im B \ssi hermite (A|B) \sim A$$
	Autrement dit, les colonnes de $B$ sont des combinaisons linéaires des colonnes de $A$ et
	donc, en particulier, il existe $C \in \M_2$ telle que $B = AC$.

\end{proof}

\begin{proposition}\label{pi_imh}
	Soit $\sapp{\pi}{\Z^2}{\ZZnZ}$ et soit $H \in \M_2$ une matrice de Hermite de la forme
	$ H = \matsqr{\alpha}{0}{\beta}{\gamma}$.
	$$H \in M \ssi \im(H)/\pmZpmZ = \pi(\im H) =
		\gen{\pi\vectcolsqr{\alpha}{\beta}, \pi\vectcolsqr{0}{\gamma}}$$
\end{proposition}

\begin{proof}
	Cela découle des deux propositions précédentes en prenant $B = \matsqr{p^m}{0}{0}{p^m}$

\end{proof}

\begin{example}[Cas pour $n = 2$]
	Nous avons $n = 2 = 2^1$ donc $m = 1$.\\
	Calculons les matrices appartenant à $$M_0 = \Set*{
			\begin{pmatrix}
				p^a & 0   \\
				j   & p^b
			\end{pmatrix}
		}{
			\begin{aligned}
				 & (a, b) \in A_0  \\
				 & 0 \le j < p^{b}
			\end{aligned}
		}$$
	Nous avons $A_0 = \Set*{(a,b)}{a + b \le 1} = \set{(0,0), (0,1), (1, 0)}$.
	Ainsi
	\begin{equation*}
		M_0 = \set{
			\matsqr{1}{0}{0}{1}, \matsqr{1}{0}{0}{2}, \matsqr{1}{0}{1}{2}, \matsqr{2}{0}{0}{1}
		}
	\end{equation*}
	Calculons les matrices appartenant à $$M_1 = \Set*{
			\matsqr{2^a}{0}{2j}{2^b}
		}{
			\begin{aligned}
				 & (a, b) \in A_1      \\
				 & 0 \le j < 2^{b - 1}
			\end{aligned}
		}$$
	Nous avons $$A_1 = \Set*{
			(a,b)}{
			\begin{aligned}
				 & a \le 1, b \le 1 \\
				 & a + b = 2
			\end{aligned}
		} = \set{(1,1)}$$
	Ainsi
	$$M_0 = \set{\matsqr{2}{0}{0}{2}}$$
	Par la proposition \ref{pi_imh}, nous pouvons réaliser la correspondances entre
	les matrices obtenues et les sous-groupes de $\ZZ$. Nous avons donc
	\begin{equation*}
		\begin{aligned}
			 & \matsqr{1}{0}{0}{1}\longrightarrow (Z/2\Z)^2              \\
			 & \matsqr{1}{0}{0}{2}\longrightarrow \gen{ \bar 1, \bar 0}  \\
			 & \matsqr{1}{0}{1}{2}\longrightarrow \gen{ \bar 1,  \bar 1} \\
			 & \matsqr{2}{0}{0}{1} \longrightarrow \gen{ \bar 0, \bar 1} \\
			 & \matsqr{2}{0}{0}{2} \longrightarrow \gen{ \bar 0, \bar 0}
		\end{aligned}
	\end{equation*}
\end{example}

\begin{proposition}\label{card_h}
	soit $H = \matsqr{\alpha}{0}{\beta}{\gamma}$ une forme normale d'Hermite. Alors
	$$\left\lvert Z^2/\im H \right\rvert = \det(H) = \alpha\gamma$$
\end{proposition}

\begin{proof}
	Montrons que $E = \Set*{(x,y)}{0 \le x \le \alpha, 0 \le y \le \gamma}$ est un système de représentants
	des classes modulo $\im H$.\\
	Pour cela, montrons que si $(x,y)$ est dans $\Z^2$, alors il est en relation
	avec un élément de $E$, et les éléments de $E$ ne sont pas en relation.\\
	Soit $(x,y) \in \Z^2$. Commençons par faire la division euclidienne de $x$ par $\alpha$ :\\
	$$ \ex q \cma 0 \le r < \alpha \cma x = \alpha q + r$$
	Nous avons $(r,y - \beta q) = (x,y) - q(\alpha, \beta)$.\\
	Faisons maintenant la division euclidienne de $y - \beta q $ par $\gamma$ :
	$$\ex t \cma 0 \le s < \gamma \cma y - \beta q = \gamma t + s$$
	Il suit que
	$$(r,s) = (x,y)- q(\alpha, \beta) - t(0, \gamma)$$
	Montrons que deux éléments de $E$ ne sont pas en relations.\\
	Soit $(x,y) \in E$. Comme $0 \le x < \alpha$, la division euclidienne de $x$ par $\alpha$ est $x$.\\
	De même, la division euclidienne de $y$ par $\gamma$ est $y$.\\
	Donc la seule relation possible d'un élément de $E$ est avec lui-même.\\
	Le nombre de classes est donc
	$$ |E| = \alpha \gamma$$

\end{proof}

\begin{remark}
	En particulier, si $H \in M$, $ord(\pi(H)) = p^{a+b}$
\end{remark}

\begin{proposition}
	Soit $H = \matsqr{p^a}{0}{jp^k}{p^b}$ une forme normale de Hermite.\\
	Alors le sous-groupe $G$ de $\ZZ$ isomorphe à $\im(H)/\pmZpmZ$ est de cardinal
	$$|G|= p^{2m - a - b}$$
\end{proposition}

\begin{proof}
	Par la proposition \ref{card_h}, on a
	$$|\Z^2/\im H| = p^{a+b}$$.


	Par le troisième théorème d'isomorphisme, on a
	\begin{equation}\label{3e_card}
		(\Z^2/\pmZpmZ)/(\im H/\pmZpmZ) \isom \Z^2/\im H
	\end{equation}
	Le sous-groupe $G$ correspondant à $H$ est $\im H/\pmZpmZ$.\\
	En prenant le cardinal de l'égalité (\ref{3e_card}), nous obtenons
	\begin{equation*}
		|G| = |\Z^2/\pmZpmZ|/|\Z^2/\im H |
	\end{equation*}
	Or, $\ZZpmZ \isom \ZZpm$, d'où
	$$|G| = p^{2m - a - b}$$
\end{proof}
%- - - - -- - - - - -- - - - - - -- - - - - - - -- -- - - - - - - - - - - - - -
\newpage
\subsection{Énumération des sous-groupes}
Grâce aux résultats démontrés dans la section précédente, nous pouvons donc énumérer\\
les sous-groupes de $\ZpmZ$. Cela nous donne le théorème suivant :
\begin{theorem}
	Soit
	\app{\psi}{\N^2}{\N}{(p,n)}{\sum_{i=0}^{n}(n-i)p^i + \sum_{i = 0}^{n}\frac{1- p^{n-i+1}}{1 - p}}
	Alors, le nombre de sous groupe de $\ZZpm$ est $\psi(p,m)$
\end{theorem}

\begin{proof}
	Par le corollaire \ref{union_m}, nous savons que M = $\bigsqcup\limits_{k = 0}^{m} M_k$
	est l'ensemble des matrices dont les colonnes engendrent les sous-groupes de $\ZZpm$.
	Ainsi, le nombre de sous-groupes de $\ZZpm$ est $|M|$.
	De plus,
	$$|M| = |\sum_{k=0}^{m} M_k| = \sum_{k=0}^{m}|M_k|$$
	\noindent
	Calculons donc le cardinal de $M_0$ dans un premier temps, puis nous calculerons
	le cardinal de $M_k$ pour $1 < k < m$.\\\vspace*{1cm}\\
	Nous avons $$M_0 =\Mz \text{ avec } A_0 = \Az$$
	Or, pour chaque couple $(a,b) \in A_0$, nous avons $p^b$ matrices différentes.
	De plus, $a \le m - b$. Donc, en remplaçant $b$ par $i$, nous obtenons
	$$ |M_0| = \sum_{i=0}^{m}(m-i)p^i$$
	\\\vspace*{1cm}\\
	Calculons maintenant le cardinal des $M_k$. Nous avons
	$$M_k = \Mk \text{ avec } A_k = \Ak$$
	Tout d'abord, pour tout couple $(a,b) \in A_k$, il y a $p^{ b-k}$ matrices différentes.\\
	De plus, pour tout ($a,b) \in A_k$, nous avons
	$a + b = m + k \implies b = m - a +k$. De cela, nous en déduisons
	\begin{equation*}
		\begin{aligned}
			         & 0 \le m - a + k \le m                             \\
			\implies & 0\le m - a \le m - k \text{, car  } 0 \le a \le m
		\end{aligned}
	\end{equation*}
	D'où posant $j = m - a$, nous avons
	\begin{equation*}
		\begin{split}
			|M_k| &= \sum_{j}^{m - k}p^{b-k}\\
			&= \sum_{j}^{m - k}p^{m - a + k - k}\\
			&= \sum_{j}^{m - k}p^{j}\\
			&= \frac{1 - p^{m-k+1}}{1-p}
		\end{split}
	\end{equation*}
	Ainsi, en remplaçant $k$ par $i$, nous en déduisons le cardinal de $M$:
	\begin{equation*}
		\begin{split}
			|M \backslash M_0| &= \sum_{i=1}^{m}|M_i|\\
			&= \sum_{i=0}^{m}\frac{1 - p^{m-i+1}}{1-p}
		\end{split}
	\end{equation*}
	D'où
	$$|M| = \sum_{i=0}^{n}(m-i)p^i + \sum_{i = 0}^{m}\frac{1- p^{m-i+1}}{1 - p}$$
	Enfin, il ne nous reste plus qu'à poser
	\app{\psi}{\N^2}{\N}{(p,n)}{\sum_{i=0}^{n}(n-i)p^i +%
		\sum_{i = 0}^{n}\frac{1- p^{n-i+1}}{1 - p}}
\end{proof}

\begin{proposition}
	Soit $n = \prod\limits_{i = 1}^k p_i^{\alpha_i}$ avec $p_i$ des nombres premiers distincts\\
	Le nombre total de sous-groupes de $\ZZ$ est
	$$\prod_{i = 0}^{k} \psi(p_i,\alpha_i)
		= \prod_i^k\mlarge[3](\sum_{l=0}^{\alpha_i}(\alpha_i-l)p_i^l +%
		\sum_{l = 0}^{\alpha_i}\frac{1- p_i^{\alpha_i-l+1}}{1 - p_i}\mlarge[3])$$
\end{proposition}

\begin{proof}
	Nous avons montré dans la section \ref{theoreme_chinois} que
	$$(\ZZ) \isom \prod_i^k(\Z/p_i^{\alpha_i}\Z)^2$$
	Par le théorème précédent, $|S((\Z/p_i^{\alpha_i}\Z)^2)| = \psi(p_i, \alpha_i)$ avec
	$S((\Z/p_i^{\alpha_i}\Z)^2)$ l'ensemble des sous-groupes de $(\Z/p_i^{\alpha_i}\Z)^2$.\\
	Ainsi,
	\begin{equation*}
		\begin{split}
			|S(\ZZ)| &= |S(\prod_i^k(\Z/p_i^{\alpha_i}\Z)^2)|\\
			&= \prod_i^k|S((\Z/p_i^{\alpha_i}\Z)^2)\\
			&= \prod_i^k\psi(p_i,\alpha_i)\\
			&= \prod_i^k\mlarge[3](\sum_{l=0}^{\alpha_i}(\alpha_i-l)p_i^l +%
			\sum_{l = 0}^{\alpha_i}\frac{1- p_i^{\alpha_i-l+1}}{1 - p_i}\mlarge[3])
		\end{split}
	\end{equation*}
\end{proof}

\begin{remark}
	\hfill
	\begin{itemize}
		\item Si $n = 0$, alors $\ZZ \isom Z^2$ et il y a une infinité de sous-groupes.
		\item Si $n = 1$, alors $\ZZ$ est le groupe trivial.
	\end{itemize}

\end{remark}
%preuve sur le calcul du nombre de sous-groupe
%%%%%%%%%%%%%%%%%%%%%%%%%%%%%%%%%%%%%%%%%%%%%%%%%%%%%%%%%%%%%%%%%%%%%%%%%%%%%%
\newpage
\section{Génération du treillis}
%Algorithme génération du treillis
%Choix de .dot et graphivz

%%%%%%%%%%%%%%%%%%%%%%%%%%%%%%%%%%%%%%%%%%%%%%%%%%%%%%%%%%%%%%%%%%%%%%%%%%%%%%%%
\newpage
\section{Quelques résultats}
%- - - - - - - - - - - - - - - - - - - - - - - - -- - - - - - - - -- - - - - -
\subsection{Pour n = 2}
%pour n = 2
%- - - - - -- - - - - - - - - - - - - - - - - - - - - - - - -- - - - - - -- - -
\subsection{Pour n = 4}
% pour n = 4
%- - - - - - - - - - - - - - - - - - - - - - - - - - - - - - - - - - - - - - -
\subsection{Pour n = 6}
% pour n = 20

%nombre de sous-groupes + treillis .dot
%%%%%%%%%%%%%%%%%%%%%%%%%%%%%%%%%%%%%%%%%%%%%%%%%%%%%%%%%%%%%%%%%%%%%%%%%%%%%%
\newpage
\section{Bibliographie}
\begin{enumerate}

	\item[[\,1\!\!]] COSTE Michel, \textit{Algèbre linéaire sur les entiers}, Mars 2018
	\item[[\,2\!\!]] \textbf{TODO livre algo}
	\item[[\,3\!\!]] PERNET Clément, \textit{Calcul de formes normales matricielles: de
		      l'algorithmique à la mise en pratique}, Séminaire SIESTE, ENS-Lyon, 12 février 2013
	\item[[\,4\!\!]] BERHURY Grégory, \textit{Algèbre le grand combat : Cours et exercices},
	      $2^e$ édition.
	      Paris : Calvage \& Mounet, 2020. 1215 p. (Mathématiques en devenir)
\end{enumerate}
\end{document}
