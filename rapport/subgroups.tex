\documentclass[11pt]{article}
\usepackage[dvipsnames]{xcolor}
\usepackage[T1]{fontenc}
\usepackage{mathtools}
\usepackage[french]{babel}
\usepackage{amsmath,amssymb,amsthm}
\usepackage{framed}
\usepackage{lmodern}
\usepackage{utils}
\usepackage{pdfpages}
\usepackage{irif}


\begin{document}
\includepdf{title.pdf}
\tableofcontents
\newpage

\section{Introduction}
	Il est très facile de décrire tous les sous-groupes d'un groupe cyclique
	d'ordre $n$ : il y en a exactement un par diviseur positif de $n$.
	Pourtant, étonnamment, décrire tous les sous-groupes d'un groupe abélien
	est en général un problème difficile.\\
	Dans ce projet, nous nous se proposons de considérer cette question pour le groupe $\ZZ$.

	D'un point de vue théorique, nous mettrons en avant la générations et la caractérisations de \\
	sous-groupes grâce aux vecteurs colonne des matrices à coefficients entier et en particulier aux formes
	normales de Hermite. Nous montrerons aussi une formule permettant de les compter.

	D'un point de vue pratique, nous créerons un programme \textsc{OCaml} capable de générer les\\
	sous-groupes de $\ZZ$ ainsi que leur treillis à partir d'un entier donné en paramètres.
\section{Quelques simplifications du problème}
\subsection{Décomposition de $n$ en éléments irréductibles}

Nous pouvons tout d'abord simplifier le problème aux cas où $n = p^m$ avec $p$ un nombre premier
et $m \in \N$. En effet la proposition suivante nous garantie que le résultat est isomorphe
\begin{proposition}
	Soit $n = \prod\limits_i^k p_i^{\alpha_i}$, avec $p_i$ des nombres premiers, alors
	$$(\ZZ) \isom \prod_i^k(\Z/p_i^{\alpha_i}\Z)^2$$
\end{proposition}

\begin{proof}
	Soit $n = \prod\limits_i^k p_i^{\alpha_i}$. Par le théorème des restes chinois, on a
	$$ \ZnZ \isom (\Z/p_i^{\alpha_i}\Z) \x \cdots \x (\Z/p_i^{\alpha_i}\Z)$$
	En particulier,
	\begin{equation*}
		\begin{split}
			\ZZ & \isom
			(\Z/p_i^{\alpha_i}\Z) \x \cdots \x (\Z/p_i^{\alpha_i}\Z) \x (\Z/p_i^{\alpha_i}\Z) \x \cdots \x (\Z/p_i^{\alpha_i}\Z)\\
				& \isom (\Z/p_i^{\alpha_i}\Z)^2 \x \cdots \x (\Z/p_i^{\alpha_i}\Z)^2
		\end{split}
	\end{equation*}
\end{proof}

En pratique, pour décomposer en entier en facteurs irréductibles, nous avons utilisé la procédure de
\textsc{$\rho$-Pollard} pour obtenir un diviseur de $n$:
\begin{verbatim}
fonction rho_pollard P n x y k i d
    Si d <> 1:
        Retourne d
    Sinon:
        x = P(x) mod n
        d = pgcd(|y - x|, n)
        Si i = k:
            Alors Retourne rho_pollard loop P n x x 2k (i + 1) d
        Sinon Retourne rho_pollard P n x y k (i + 1) d
\end{verbatim}
Puis nous répétons la procédure jusqu'à que les diviseurs soient premier.\\
En triant et en regroupant les nombres premier, nous obtenons donc les différents $p^{\alpha_i}_i$.\\
Dans notre implémentation, $P(X) = X^2 - 1$ et $n$ n'est pas premier.
%TODO : Ajout algo test primarite ?

\newpage
\subsection{Simplification des sous-groupes}

\begin{proposition}
	$$\Z^2/n\Z \x n\Z \isom \ZZ $$
\end{proposition}
\begin{proof}
	Soit \app{\varphi}{\Z^2}{\ZZ}{(a,b)}{(\bar a, \bar b)}
$\varphi$ est surjective par définition de la classe d'équivalence de a et b.
Montrons que $\ker \varphi = n\Z \x n\Z$.

\begin{equation*}
	\begin{split}
		&(a,b) \in \ker \varphi \\
		&\text{ssi } \varphi(a,b) = (\bar 0, \bar 0)\\
		&\text{ssi } (\bar a, \bar b) = (\bar 0, \bar 0)\\
		&\text{ssi } \bar a = \bar 0 \text{ et } \bar b = \bar 0\\
		&\text{ssi } a \in n\Z \text{ et } b \in n\Z\\
		&\text{ssi } (a,b) \in  n\Z \x n\Z
	\end{split}
\end{equation*}
Ainsi par le premier théorème d'isomorphisme, on a
$$\Z^2/n\Z \x n\Z \isom \ZZ $$
\end{proof}
Ainsi le problème se résout à trouver les sous-groupes $G$ de $\Z^2$ tels que
$H = \matsqr{a}{0}{b}{c}$\\
et
$n\Z \x n\Z \subseteq G = \gen{\vectcolsqr{\bar a}{\bar b}, \vectcolsqr{0}{\bar c}}$


\section{Matrices à coefficients entier et forme normales de Hermite}
Nous avons vu dans la section précédente qu'il était possible de caractériser les sous-groupe de
$\ZZ$ par une matrice $H = \matsqr{a}{0}{b}{c}$. Cependant, ces matrices ne sont pas uniques. C'est
pourquoi, nous allons utiliser les formes normales d'Hermite.
Énonçons d'abord quelques propriété sur les matrices à coefficients entier.
%problème section précédente, il existe un nombre important de matrice similaire i.e qui
%engendre le même sous groupe
\subsection{Matrices à coefficients entier}
\begin{proposition}
	Soient $A \in \M_{m,n}(\Z)$ et $Q \in \GL_n(\Z)$, alors
	$$\im AQ = \im A$$
\end{proposition}
\begin{proof}
	Soit $y \in \im AQ$, il existe $x \in \Z^n$ tel que $y = AQx$. Or,
	\begin{align*}
		&y = AQx\\
		\implies &y = A(Qx)\\
		\implies &y \in \im A
	\end{align*}
	Donc $\im AQ \subseteq \im A$.\\
	Soit $y \in \im A$. Il existe $x \in \Z^n$ tel que $y = Ax$.\\
	Cherchons $z \in \Z^n$ tel que $y = Ax = AQz$
	\begin{align*}
		&Ax = AQz\\
		\implies &A(x) = A(Qz)\\
		\implies &x = Qz\\
		\implies &\inv Q x = z \text{ (car $B \in \GL_n(\Z)$)}
	\end{align*}
	Donc il existe bien un $z \in \Z^n$ tel que $ABz = y$. Donc $y \in \im AQ$.\\
	D'où $\im AQ = \im A$

\end{proof}
\subsection{Formes normales de Hermite}
Nous allons désormais énoncer quelques propriétés utiles sur les formes normales de Hermite.
\begin{definition}
	Soit $A \in \M_{m,n}(\Z)$. Alors il existe une unique matrice échelonnée
	réduite suivant les colonnes $H \in \M_{m,n}(\Z)$ telle qu'il existe $Q \in \GL_n(\Z)$
	avec $H = AQ$. La matrice $H$ s'appelle la forme normale de Hermite de A.
\end{definition}
Nous supposerons l'unicité admise, l'algorithme suivant nous montre son existence.

\begin{verbatim}
	Fonction hermite A:
	    Pour chaque i <= n :


\end{verbatim}
% preuve existence ?
% algo forme hermite
% preuve AX = C solution ssi on peut annuler C ?

\section {Génération et énumération des sous-groupes}
\subsection{Génération des sous-groupes}
%preuve sur la bonne forme de forme normale de Hermite
\subsection{Énumération des sous-groupes}
%preuve sur le calcul du nombre de sous-groupe

\section{Génération du treillis}
%preuve H C H' <=> Hermite(H'|H) = H'
%Algorithme génération du treillis
%Choix de .dot et graphivz

\section {Quelques résultats}
\subsection{Pour n = 2}
%pour n = 2
\subsection{Pour n = 4}
% pour n = 4
\subsection{Pour n = 20}
% pour n = 20

%nombre de sous-groupes + treillis .dot

\section{Références}
%livre d'algo
%poly du prof

\end{document}
